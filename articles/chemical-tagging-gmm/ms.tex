%% This document is part of the snob project
%% Copyright 2017 the authors. All rights reserved.

\documentclass{aastex61}

\usepackage{bm}

% For revision history
\IfFileExists{vc.tex}{\input{vc.tex}}{
    \newcommand{\githash}{UNKNOWN}
    \newcommand{\giturl}{UNKNOWN}}

% Define commands
\newcommand{\article}{\emph{Article}}
\newcommand{\acronym}[1]{{\small{#1}}}
\newcommand{\project}[1]{\textsl{#1}}

% Have some maaaath.
\def\infinity{\rotatebox{90}{8}}

% Affiliation(s)
\newcommand{\moca}{
	\affil{School of Physics and Astronomy, Monash University, 
		   Melbourne, Clayton VIC 3800, Australia}}
\newcommand{\claytonfit}{
	\affil{Faculty of Information Technology, Monash University,
	       Melbourne, Clayton VIC 3800, Australia}}
\newcommand{\caulfieldfit}{
	\affil{Faculty of Information Technology, Monash University,
		   Melbourne, Caulfield East VIC 3145, Australia}}



\received{}
\revised{}
\accepted{}

\submitjournal{TBD} % ApJ, MNRAS, PASA?

\shorttitle{Chemical Tagging Using Minimum Message Length. I.}
\shortauthors{Casey et al.}

\begin{document}

\title{Chemical Tagging Using Minimum Message Length. I.\\
       Gaussian Mixture Modelling}

\correspondingauthor{Andrew R. Casey}
\email{andrew.casey@monash.edu}

\author[0000-0003-0174-0564]{Andrew R. Casey}
\moca
\claytonfit

\author{(some order of:}

\author[0000-0003-2952-859X]{John Lattanzio}
\moca

\author[0000-0002-0583-5918]{David Dowe} 
\claytonfit

\author[0000-0002-1716-690X]{Aldeida Aleti}
\caulfieldfit

\author{)}

\author{others?}
 
\begin{abstract}
Chemical tagging seeks to identify co-natal stars by their present-day 
photospheric abundances, long after their phase space similarity is lost.
In principle the detailed chemical abundances of a transformative number of
stars could be used to reconstruct the evolution of the Milky Way, providing
insight on supernovae yields, galactic dynamics (e.g., constraints on radial 
mixing), the initial mass function, as well as the formation and destruction
of star clusters. Some progress has been made towards chemical tagging, but 
here we argue that even the \emph{extent} of the problem has not been realised.
We introduce the Real World problems associated with chemical tagging at scale,
and describe how currently considered methods will inevitably fail even under
the most unrealistically optimistic conditions. We introduce the minimum message
length to the astronomical community as a promising alternative to chemical
tagging, one which has the capability to address all ensivaged problems with
chemical tagging simultaneously. We perform chemical tagging experiments with
infinte Gaussian mixture models, and demonstrate using real and generated data 
that MML outperforms all other considered penalty techniques. 
\end{abstract}

\keywords{}

\section{Introduction} 
\label{sec:introduction}

\citet{Freeman_2002} introduced chemical tagging as an idea to identify groups 
of stars that formed together in the same gas cloud, which are no longer
identifiable in phase space, either by their physical location or orbital
dynamics.  This idea is attractive because the observable chemical abundances of 
stars remains largely unchanged throughout a star's lifetime, whereas phase 
space information is quickly damped through dynamical interactions. Moreover, 
while most stars are thought to form in star clusters \citep{someone}, only the
most massive or isolated star clusters are able to survive the star formation
process \citep{someone2}. For this reason we rely on `field' stars -- that is,
stars that are not clearly associated with a known star cluster -- in order to
infer the formation of the Milky Way.


Some progress has been made towards chemical tagging in simplistic experiments, 
but the full extent of chemical tagging has yet to be realised. Indeed, in this 
\article\ we assert that even \emph{the full extent of the problem} of chemical 
tagging has yet to be realised. Before reviewing the existing literature on
what has been done with chemical tagging, we will outline the necessary
components to perform chemical tagging.


\subsection{The Problem: What's needed for chemical tagging?}
\label{section:the-problem}

Given a set of stars that have been observed spectroscopically, and detailed
chemical abundances have been derived, the following issues will arise when
trying to perform chemical tagging:

\begin{enumerate}
\item The number of star clusters (or star-forming events) where at least one
      star was observed and detailed chemical abundances were derived.
\item The number of latent factors (e.g., supernovae yields, AGB yields).
      Within the chemical tagging literature, this is commonly referred to as
      the number of `effective chemical dimensions'.
\item The multivariate factor loads of individual latent factors. For example,
      if core-collapse supernovae and AGB stars had both contributed to the
      chemical abundances of all stars in the sample, then the
      \emph{factor loads} would be the \emph{yield} of the supernovae, or the
      yield of relative abundances that were produced by the AGB star. Each
      star will be affected differently by these yields (see next item).
\item The relative \emph{scoring} of different latent factors on each star.
      This is a scaling that is applied to the latent yields. Continuing our
      analogy on yields from core-collapse supernovae and AGB stars, the
      \emph{factor scores} can be thought of as the relative contributions of
      the different factor loads, for a single star.
\item The relative weighting (yet another latent factor, which is however not
      common between all mixtures) of individual mixtures.
\item The multivariate mean abundances and covariance matrices of each star
      cluster, where the covariance matrix includes off-diagonal terms that
      describe the \emph{data} correlations (e.g., the correlations that result
      from trying to measure multiple abundances from the same data).
\item The intrinsic \emph{specific variances} of the individual abundances.
\item Atomic diffusion, veiling, and thermohaline mixing: The surface chemical
      abundances of stars \emph{will change} over its lifetime.
\item Probabilistic priors (perhaps very hard priors) on joint responsibilities
      of stars that are specified from astrometric constraints.
\item There are missing data. 
\item There are \emph{upper limits} on some data.\footnote{
        Spectroscopists are probably the worst in the Universe when it comes to
        reporting upper limits. They will report an `\emph{upper limit}', when
        what might mean is the \emph{absolute limit} where there is zero
        likelihood that the true value is above the given point estimate (which might as well be $+\infinity$, or
        they may describe a \emph{pseudo-1$\sigma$} limit which is estimated
        from the locally (yet covariant!) heteroskedastic noise in the data (which often might as well be $\approx$0).
        In reality, an upper limit ought to be given as a probability density
        function that is skewed in one direction, perhaps similar to the shape
        of a log-normal distribution. In any case, the confidence to which they
        describe their `\emph{upper limit}' ought to be given.}
\item There are correlations in the measurements that result from trying to
      measure two (or more) abundances from the same data. As an example,
      consider a rare -- yet important -- chemical abundance that has a single
      atomic absorption line which is blended with a more common elemental
      abundance. In all likelihood, there will be a strong positive correlation
      between the abundance derived between the rare and more common elemental
      abundance.
\end{enumerate}


Previous approaches have tried to solve separate components of chemical tagging (e.g. clustering). Some notes on this.

These approaches are not consistent, because the problems are intertwined (e.g., clustering, latent factors).


% What work has been done on chemical tagging?

% Unfortunately these approaches have only sought to solve individual problems
% within chemical tagging. Doing so -- and then later using those results to
% infer something else about the galaxy -- makes both results inconsistent,
% both statistically, and otherwise.



In Section \ref{sec:model} we introduce a simplistic approach to chemical
tagging that relies on probabilistically modelling stellar chemical abundances
as a mixture of (potentially infinite) multivariate Gaussian distributions.



In this \article\ we will introduce a simple probabilistic approach to
chemical tagging, where the data are modelled as a mixture of multivariate
Gaussian distributions (Section \ref{sec:the-model}). In Section \ref{sec:mml}
we introduce the principle of minimum message length (MML), and formulate our
objective function by minimising the total message length. Our model is
simplistic in the sense that we will ignore many of the self-consistency
issues already introduced. However, in Section \ref{sec:experiments} we show
that our MML-based approach is superior to all other methods considered for
chemical tagging, all of which cannot alleviate the self-consistency issues.
In Section \ref{sec:discussion} we will describe how MML can overcome these
aforementioned issues, in order to construct a fully-consistent probabilistic
approach to chemical tagging.



\section{The Model}
\label{sec:the-model}

We make the following explicit assumptions:

\begin{itemize}
\item We assume that the data (e.g., the $D$ detailed chemical abundances of 
      $N$ stars) can be represented by a mixture of $K$ multivariate Gaussian
      distributions.
\item We assume that there are no missing data values. For example, if a star
      has been observed, then there is a complete set of $D$ chemical 
      abundances available.
\item We will assume that the number of \emph{true} multivariate Gaussian
      distributions $K_{true}$ is not known. This assumption applies to our 
      experiements involving generated data, and those using real data.
\item In addition to not knowing the \emph{number} of multivariate normal
      distributions, we further assume that the means $\bm{\mu}$ and 
      covariance matrices $\bm{C}$ of each $K$th mixture is unknown.
\item We assume that the covariance matrix of individual components can be
      described as `free' or `full', insomuch that the data within a cluster
      can have off-diagonal correlation terms, and those terms are unknown.
\item In all experiments we will assume that the data have homoskedastic noise
      properties.
\item The relative weights $\bm{w}$ (or mixing proportions) of $K$ 
      distributions is unknown. 
\item We assume a relative (or conditional) probability distribution for each 
      star belonging to a given cluster. That is to say that we do not adopt a 
      `\emph{hard selection}' approach to cluster modelling, where stars would
      be assigned as definitively belonging to one cluster or another. Our
      approach could be described as `\emph{soft} clustering'. This assumption
      has practical numerical outcomes and implications for the message length.
\end{itemize}

We have made relatively few explicit assumptions about the model for the data,
but even the model described constitutes a very formidable model selection 
problem. This is because we do not know \emph{how many} multivariate Gaussian
distributions we ought to chose to fit the data. Including more mixtures will
consistently provide a better fit to the data (in a likelihood sense), but we 
must make some decision or heuristic as to whether the additional mixture is 
justified.

The probability density function $f$ for a multivariate normal distribution 
with $d$ dimensions is given by

\begin{equation}
    f(\bm{y}|\bm{\mu},\bm{\Sigma}) 
        = \frac{1}{\sqrt{(2\pi)^d|\bm{\Sigma}|}}
          \exp{\left[-\frac{1}{2}(\bm{y} - \bm{\mu})^\intercal\bm{\Sigma}^{-1}(\bm{y} - \bm{\mu})\right]}
\end{equation}

\noindent{}where $\bm{\mu}$ and $\bm{\Sigma}$ is the multivariate mean and 
covariance matrix, and $\bm{y}$ are the data.  

% Full probability for the data.


When modelling a mixture of
Gaussians, there are two approaches in determining assignments (or
responsibilities) of each star to a given cluster. One approach is to include
a (discrete) model parameter for every data point, which represents the index
of the mixture that the observation is assigned to. The second approach is to 
use a weighted mixture. These two are mathematically identical.
% TODO


\begin{equation}
% TODO
\end{equation}


\subsection{Minimum Message Length}
\label{sec:mml}

Background to MML, etc.

We want to minimize the message length.

Link to Bayesian evidence, other things, etc.

In order to minimize the full message length, we must specify the priors on
our parameters.


\subsection{Prior probabilities on the parameters}


We assume a uniform prior on the mixing weights $\bm{w}$, only requiring that
$\sum_{k=1}^{K}w_k = 1$. A useful extension of this work might be to impose
a Dirchlet prior on the mixing weights $\bm{w}$, or a distribution inspired by
the stellar mass function and observed selection function.


We adopt an improper uniform prior of 
$\mathcal{U}(\bm{\mu}) = [-\infinity, \infinity]$ on all multivariate mean 
abundances $\bm{\mu}$ (in all $D$ dimensions). This prior has no impact on
our inferences and becomes proper when it enters the (accepted) posterior
distribution. We similarily could adopt a proper and large uniform prior on all 
mean abundances of $\mathcal{U}(\bm{\mu}) = [-12, 12]$ and the result would be
the same.

We adopt a congujate inverted Wishart prior for the covariance matrices in the
individual mixtures. The joint prior density for the parameters of a single
mixture is given by the limiting form of the normal inverted-Wishart
density \citep[e.g., Section 5.2.3 of ][]{Schafer_1990}:


\begin{equation}
    p(\bm{\mu},\bm{\Sigma}) \propto |C|^{\frac{d + 1}{2}} \quad .
\end{equation}



\subsection{Expected Fisher information}

Computing the expected Fisher information requires the second order partial
derivatives of the negative log-likelihood $-\mathcal{L}(\bm{y}|\bm{\mu},\bm{\Sigma})$.
This can be tricky for mixtures -- explain why.
We approximate the determinant of of the Fisher information matrix,
$|\mathcal{F}(\bm{\mu},\bm{\Sigma})|$ as the product of $|\mathcal{F}\left(\bm{\mu}\right)|$ and $|\mathcal{F}\left(\bm{\Sigma}\right)|$ \citep{Oliver_1996,Roberts_1998}.

% Equation for log-likelihood of the data:
% \begin{equation}
% \mathcal{L}\left(\bm{y}|\bm{\mu},\bm{\Sigma}) = -\frac{Nd}{2}\log{\left(2\pi\right)} - \frac{N}{2}\log{|\bm{\Sigma}|} - \frac{1}{2}\Sigma_{i=1}^{N}(\bm{y}_i - \bm{mu})^\intercal{}\Sigma^{-1}(\bm{y}_i - \bm{\mu})
%\end{equation}

Taking the second derivative of the log-likelihood of the data with respect to $\bm{\mu}$:

\begin{equation}
\frac{\delta\mathcal{L}}{\delta\bm{\mu}} = \Sigma_{i=1}^{N}\bm{C}^{-1}(\bm{y}_i - \bm{\mu})
\end{equation}

\begin{equation}
\frac{\delta^2\mathcal{L}}{\delta\bm{\mu}^2} = N\bm{C}^{-1}
\end{equation}


Computing the Fisher information of the covariance matrix $|\mathcal{F}\left(\bm{C}\right)|$ is harder. For a full covariance matrix with off-diagonal terms, the total
number of free parameters required is $d(d+1)/2$.
\citep{Magnus_1988} derived an analytical expression:
% Dwyer_1967, Bozdogan_1990? Drton_2009?
% TODO: This derivation is a nightmare to find


\begin{equation}
|\mathcal{F}\left(\bm{C}\right) = N^\frac{d(d+1)}{2}2^{-d}|\bm{C}|^{-(d+1)}
\end{equation}

Giving the approximate determinant of the Fisher information $|\mathcal{F}\left(\bm{\mu},\bm{C}\right)|$ as:

\begin{eqnarray}
  |\mathcal{F}\left(\bm{\mu},\bm{C}\right)| & \approx & |\mathcal{F}\left(\bm{\mu}\right)|\cdot|\mathcal{F}\left(\bm{C}\right)| \\
  & \approx & N^{d}|\bm{C}|^{-1}N^\frac{d(d+1)}{2}2^{-d}|\bm{C}|^{-(d+1)} \\
  |\mathcal{F}\left(\bm{\mu},\bm{C}\right)| & \approx & N^\frac{d(d+3)}{2}2^{-d}|\bm{C}|^{-(d+2)}
\end{eqnarray}





A common method to estimate the parameters of a Gaussian distribution is by
maximum likelihood, where the log-likelihood $\mathcal{L}$ is maximized. In
order to do this, one computes the gradient of the log-likelihood function
with respect to the parameters and solves the resulting equations.


- Encoding the message.


- Process of expectation and maximization for each step.


\section{Experiments}
\label{sec:experiments}

Here we describe simple chemical tagging experiments that are so optimistic
that they could be considered laughable, but ones which will still fail for
even the most simplistic chemical tagging approaches.

Each experiment will edge closer to a representative example of chemical
tagging.

\subsection{Experiment 0: Position and Velocity information}

\subsection{Experiment 1: Using clusters, full complement of chemical abundances}

\subsection{Experiment 2: Using all clusters, full complement of chemical abundances, random fractions of field stars}

\subsection{Experiment 3: Using clusters, limited set of chemical abundances}

\subsection{Experiment 4: Fake data to replicate clusters, but use latent factor}

\subsection{Experiment 5: Fake data to replicate clusters, but use multiple latent factors}


\section{Discussion}
\label{sec:discussion}


\acknowledgments

This research is supported by an Australian Research Council Discovery Project
grant (DP160100637). 
Source code for this project is available at \texttt{\giturl}. This document
was compiled on \gitdate\ from revision hash \texttt{\githash} in that
repository. 


\software{astropy \citep{Robitaille:2013}, numpy, scipy, scikit-learn} 

\begin{thebibliography}{}

\bibitem[Astropy Collaboration et al.(2013)]{Robitaille:2013} Astropy Collaboration, Robitaille, T.~P., Tollerud, E.~J., et al.\ 2013, \aap, 558, A33 

\end{thebibliography}


\end{document}
