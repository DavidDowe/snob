\documentclass{article}

% if you need to pass options to natbib, use, e.g.:
% \PassOptionsToPackage{numbers, compress}{natbib}
% before loading nips_2017
%
% to avoid loading the natbib package, add option nonatbib:
% \usepackage[nonatbib]{nips_2017}

%\usepackage{nips_2017}

% to compile a camera-ready version, add the [final] option, e.g.:
\usepackage[final]{nips_2017}

\usepackage[utf8]{inputenc} % allow utf-8 input
\usepackage[T1]{fontenc}    % use 8-bit T1 fonts
\usepackage{hyperref}       % hyperlinks
\usepackage{url}            % simple URL typesetting
\usepackage{booktabs}       % professional-quality tables
\usepackage{amsfonts}       % blackboard math symbols
\usepackage{nicefrac}       % compact symbols for 1/2, etc.
\usepackage{microtype}      % microtypography

\usepackage{amssymb} % for \intercal
\usepackage{bm}      % for bold vectors
\newcommand{\vect}[1]{\boldsymbol{\mathbf{#1}}}
\def\veccov{\vect{C}}
\def\vecmean{\vect{\mu}}
\def\vectheta{\vect{\theta}}
\def\weight{w}
\def\weights{\vect{\weight}}
\def\datum{y}
\def\data{\vect{\datum}}
\def\likelihood{\mathcal{L}}
\def\current{^{\mathrm{(K)}}}
\def\future{^{\mathrm{(K} + \Delta\mathrm{K)}}}

\title{
  Unsupervised learning of Gaussian mixture models in linear time
  using a variational Bayesian method
}

% The \author macro works with any number of authors. There are two
% commands used to separate the names and addresses of multiple
% authors: \And and \AND.
%
% Using \And between authors leaves it to LaTeX to determine where to
% break the lines. Using \AND forces a line break at that point. So,
% if LaTeX puts 3 of 4 authors names on the first line, and the last
% on the second line, try using \AND instead of \And before the third
% author name.

\author{
  Andrew R.~Casey\\
  Monash University\\
  \texttt{andrew.casey@monash.edu}\\
  \And 
  Aldeida Aleti\\
  Monash University\\
  \texttt{aldeida.aleti@monash.edu}\\
  \AND
  David Dowe\\
  Monash University\\
  \texttt{david.dowe@monash.edu}\\
  \And
  John C.~Lattanzio\\
  Monash University\\
  \texttt{john.lattanzio@monash.edu}\\
}

\begin{document}
% \nipsfinalcopy is no longer used

\maketitle

\begin{abstract}
  The appropriate number of components in a Gaussian mixture model is usually 
  determined by adopting a heuristic to penalize the number of free parameters.
  This approach unnecessarily separates model selection from parameter estimation,
  and requires a large set of mixtures with an increasing number of components
  to be optimized.
  We formulate an objective function for a finite Gaussian mixture model using
  Minimum Message Length (MML), and derive an expression to approximate that
  objective function for mixtures with any number of components. 
  These approximations provide a probability density for the evaluated objective
  function of future mixtures that have yet to be computed, allowing us to 
  \emph{jump} towards the optimal number of components.
  We show that our approach can simultaneously perform parameter estimation and 
  model selection for multivariate Gaussian mixtures in linear time.
\end{abstract}

\section{Introduction}
% Gaussian mixture modelling is a common problem.
% When the number of mixtures is unknown, the problem is normally 
% attacked by optimizing multiple different models, and performing
% model selection after-the-fact.
% Model selection is typically governed by a penalized heuristic,
% e.g., BIC, AIC, etc.

% Introducing MML
% Different approaches in terms of where to start from:
% Some things about how MML has been used to grow
% Some example where a mixture has been used where K=N,
% because merging/deleting components is computationally
% more efficient than splitting.

% In these approaches, the long-term memory of previously
% considered mixtures is usually discarded.
% Here we will outline an objective function using MML
% We will show that the difference in the message length
% between the current state and future unobserved states
% can be approximated.

\section{The objective function}
\label{sec:objective-function}


\noindent{}The probability density function $f$ for a multivariate normal 
distribution with $D$ dimensions is,

\begin{equation}
  f(\data|\vecmean,\veccov) 
      = \frac{1}{\sqrt{(2\pi)^d|\veccov|}}
        \exp{\left[-\frac{1}{2}(\data - \vecmean)^\intercal\veccov^{-1}(\data - \vecmean)\right]}
\end{equation}

\noindent{}where $\vecmean$ and $\veccov$ is the multivariate mean and 
covariance matrix, and $\data$ are the data.
For a fixed number of $K$ multivariate Gaussian mixtures, the probability
density function is given by:

\begin{equation}
  f(\data|\vectheta) = \sum_{k=1}^{K} \weight_k \, f(\data|\vectheta_k)
\end{equation}

\noindent{}Where $\vectheta_k \equiv \{\vecmean_k, \veccov_k\}$,
$\vectheta \equiv \{\vectheta_1,\dots,\vectheta_K,\weight_1,\dots,\weight_K\}$, 
and $\weight_k$ is the relative weight of the $k$-th mixture such that
$\sum_{k=1}^{K}\weight_k = 1$.


\subsection{Minimum Message Length}
\label{sec:mml}

The Minimum Message Length \citep[MML;][]{Wallace_1968} principle is a formal
description of Occam's razor that is based on information theory.
% Taking Bayes and Shannon.
% TODO
The MML principle states that the best explanation of the data is the model
that describes the data using the smallest amount of information (shortest
message), where information is described as per \citet{Shannon_1948}. 


Calculating the optimal lossless message length of a complex model can be
non-trivial or intractable. Here we make use of the \citet{Wallace_1987}
approximation of the message length:

\begin{equation}
\label{eq:message-length}
  I(\vectheta,\data) 
    = \frac{Q}{2}\log\kappa{\left(Q\right)} 
    - \log{\left(\frac{p(\vectheta)}{\sqrt{|\mathcal{F}(\vectheta)|}}\right)}
    - \likelihood\left(\data|\vectheta\right) 
    + \frac{Q}{2} \quad .
\end{equation}

Here $p(\vectheta)$ is the prior, $Q$ is the number of free parameters in the
model, $\kappa\left(Q\right)$ is a function to approximate the lattice
quantisation constant for $Q$ free parameters \citep[e.g.,][]{Conway_1984}, 
and $|\mathcal{F}\left(\vectheta\right)|$ is the determinant of the 
\emph{expected} Fisher information matrix (the second-order partial 
derivatives of the negative log-likelihood function 
$-\likelihood(\data|\vectheta)$).


The MML principle requires that we encode everything required to reconstruct
the message, including our prior beliefs on the model parameters.The message 
for a finite Gaussian mixture must therefore encode:
\begin{enumerate}
  \item The number of Gaussian mixtures, $K$.
  \item The relative weights $\weights$ of the $K - 1$ Gaussian mixtures. Only
        $K - 1$ must be encoded because $\sum_{k=1}^{K}\weight_k = 1$.
  \item The component parameters $\vecmean$ and $\veccov$ for all $K$ Gaussian
        mixtures.
  \item The data, given the model parameters.
  \item The lattice quantisation constant $\kappa(Q)$ for the number of model
        parameters $Q$.
\end{enumerate}

We provide the message length of each component in turn before outlining our
objective function.  First, we assume a prior on $K$ of $p(K) \propto 2^{-K}$
such that the length of the optimal lossless message to encode $K$ is 
$I(K) = -\log{p(K)} = K\log{2} + \textrm{constant}$.  Given a uniform prior
on $\weights$ and the requirement that $\sum_{k=1}^{K}\weight_k = 1$, the 
weights can be treated as parameters of a multinomial distribution, which can
be optimally encoded in a message of length:
\begin{equation}
  I(\weights) 
    = \frac{K - 1}{2}\log{N} 
    - \frac{1}{2}\sum_{k=1}^{K}\log\weight_k 
    - \log{\Gamma{\left(K\right)}} \quad .
\end{equation}

In order to properly encode the mixture parameters $\vecmean$ and $\veccov$
for all $K$ mixtures, we must encode both our prior belief on those parameters,
and the determinant of the expected Fisher information matrix. For the $k$-th
mixture this becomes,

\begin{equation}
  I(\vecmean_k,\veccov_k) = -\log{\left(\frac{p(\vecmean_k,\veccov_k)}{\sqrt{|\mathcal{F}\left(\vecmean_k,\veccov_k\right)|}}\right)}
                          = -\log{p(\vecmean_k,\veccov_k)} + \frac{1}{2}\log{|\mathcal{F}\left(\vecmean_k,\veccov_k\right)|}
\end{equation}

\noindent{}and for all mixtures:

\begin{equation}
  I(\vecmean,\veccov) = -\sum_{k=1}^{K}\log{p(\vecmean_k,\veccov_k)} + \frac{1}{2}\sum_{k=1}^{K}\log{|\mathcal{F}\left(\vecmean_k,\veccov_k\right)|} \quad .
  \label{eq:I_component_params}
\end{equation}

We adopt an improper uniform prior of $\mathcal{U}(\vecmean) = [-\infty, +\infty]$ 
on $\vecmean$, and a conjugate inverted Wishart prior for the covariance matrix
of individual mixtures \citep[e.g., Section 5.2.3. of ][]{Schafer_1997}:
\begin{equation}
  p(\vecmean_k, \veccov_k) \propto |\veccov_k|^{\frac{D+1}{2}} \quad .
  \label{eq:covariance-prior}
\end{equation}

We approximate the determinant of the Fisher information 
$|\mathcal{F}(\vecmean_k, \veccov_k)|$ as the product of 
$|\mathcal{F}\left(\vecmean_k\right)|$ and $|\mathcal{F}\left(\veccov_k\right)|$ 
\citep{Oliver_1996,Roberts_1998}.  Taking the second derivative of the
log-likelihood function $\likelihood\left(\data|\vecmean,\veccov\right)$,
with respect to $\vecmean$ yields,
\begin{equation}
  |\mathcal{F}\left(\vecmean_k\right)| = (N\weight_k)^{D}|\veccov_k|^{-1} \quad .
\end{equation}

We make use of an analytical expression for the determinant of the Fisher
information for a covariance matrix $|\mathcal{F}(\veccov_k)|$ \citep{Dwyer_1967,Magnus_1988,Kasarapu_2015},
\begin{equation}
  |\mathcal{F}\left(\veccov_k\right)| = (N\weight_k)^\frac{D(D+1)}{2}2^{-D}|\veccov_k|^{-(D+1)} \quad ,
\end{equation}

\noindent{}giving the following approximation for the determinant of the
expected Fisher information matrix,
\begin{eqnarray}
  |\mathcal{F}(\vecmean_k,\veccov_k)| & \approx & |\mathcal{F}(\vecmean_k)|\cdot|\mathcal{F}(\veccov_k)| \nonumber \\
  |\mathcal{F}(\vecmean_k,\veccov_k)| & \approx & (N\weight_k)^{D}|\veccov_k|^{-1}(N\weight_k)^\frac{D(D+1)}{2}2^{-D}|\veccov_k|^{-(D+1)} \nonumber \\
  |\mathcal{F}(\vecmean_k,\veccov_k)| & \approx & (N\weight_k)^\frac{D(D+3)}{2}2^{-D}|\veccov_k|^{-(D+2)} \quad .
\end{eqnarray}

\noindent{}which, with Eq. \ref{eq:covariance-prior}, we can substitute into Eq. \ref{eq:I_component_params}:

\begin{eqnarray}
I(\vecmean,\veccov) &=& -\sum_{k=1}^{K}\log{|\veccov_k|^{\frac{D + 1}{2}}}
 + \frac{1}{2}\sum_{k=1}^{K}\log{\left[(N\weight_k)^\frac{D(D+3)}{2}2^{-D}|\veccov_k|^{-(D+2)}\right]} \nonumber \\
I(\vecmean,\veccov) &=& -\frac{(D + 1)}{2}\sum_{k=1}^{K}\log{|\veccov_k|}
 + \frac{D(D+3)}{4}\sum_{k=1}^{K}\log{\left(N\weight_k\right)} -\frac{KD}{2}\log{2} -\frac{D+2}{2}\sum_{k=1}^{K}\log{|\veccov_k|} \nonumber \\
%I(\vecmean,\veccov) &=& \frac{D(D+3)}{4}\sum_{k=1}^{K}\log{\left(N\weight_k\right)} -\frac{KD}{2}\log{2} -\frac{(D + 1)}{2}\sum_{k=1}^{K}\log{|\veccov_k|}
% +-\frac{D+2}{2}\sum_{k=1}^{K}\log{|\veccov_k|} \nonumber \\
% TODO: UPDATE....
\end{eqnarray}

If we assume that the data have homoskedastic noise properties, then the 
precision of each measurement $\mathcal{E}$ relates the probability of a
datum $\textrm{Pr}(\datum_n)$ can be related to the probability density of
the datum, given the model 
$\textrm{Pr}(\datum_n) = \mathcal{E}^{D}\textrm{Pr}(y_i|\mathcal{M})$.
In the paper we adopt $\mathcal{E} = 0.01$, but we note that the value of
$\mathcal{E}$ has no effect on our inferences.  The message length of encoding
a datum given the model is,
\begin{equation}
  I(\data_n) = -\log{\textrm{Pr}(\data_n)} = -D\log\mathcal{E} - \log\sum_{k=1}^{K}\weight_{k}f_{k}(\data_n|\vecmean_k,\veccov_k)
\end{equation}

\noindent{}and for the entire data:
\begin{eqnarray}
  I(\data|\vectheta) &=& -ND\log\mathcal{E} - \sum_{n=1}^{N}\log\sum_{k=1}^{K}w_{k}f_k(\data_n|\vecmean_k,\veccov_k) \quad .
\end{eqnarray}

The final part of the message is to encode the lattice quantisation constant,
which arises from the approximation to the (so-called) strict MML, where
parameters are quantised into intervals in high dimensional space in order to
be losslessly encoded.  We use an approximation of the lattice constant
\citep[see Sections 5.1.12 and 3.3.4 of ][]{Wallace_2005} such that,
\begin{equation}
  \log\kappa(Q) = \frac{\log{Q\pi}}{Q} - \log{2\pi} - 1 \quad ,
\end{equation}

\noindent{}where $Q$ is the total number of free parameters in the model. Each
component requires $D$ parameters to specify $\vecmean$, and a covariance
matrix with off-diagonal terms requires $\frac{1}{2}D(D+1)$ parameters.
Because only $K - 1$ weights need to be encoded, the total number of free
parameters is given by ${Q = K\left[\frac{D(D+3)}{2} + 1\right] - 1}$.


\subsection{Message length for a finite mixture of multivariate Gaussian components}

The message lengths of the optimally encoded components allows us to construct
our full objective function:
\begin{equation}
  I(\vectheta,\data) = I(K) + I(\weights) + \sum_{k=1}^{K}I(\vectheta_k) + I(\data|\vectheta) + \frac{1}{2}\left(\log{Q\pi} - Q\log{2\pi}\right)\quad .
\end{equation}

\noindent{}Substituting the expressions outlined in the previous subsection
(except for $Q$, which we leave unexpanded for brevity), we arrive at the full
message length for a finite mixture of $K$ multivariate Gaussian components:
\begin{eqnarray}
  I(\vectheta,\data) &=&
      K\log{2} % I(K)
    + \frac{K - 1}{2}\log{N} - \frac{1}{2}\sum_{k=1}^{K}\log\weight_k - \log{\Gamma(K)} % I(\weights) 
    + \frac{1}{2}\sum_{k=1}^{K}\log{|\veccov_k|} \nonumber \\ % 
   &+&\frac{K}{2}\left[\frac{D(D+3)}{2}\log{N} - D\log{2}\right] % I(\vectheta_k) = I(\vecmean_k,\veccov_k)
    - ND\log\mathcal{E} - \sum_{n=1}^{N}\log\sum_{k=1}^{K}w_{k}f_k(\data_n|\vecmean_k,\veccov_k) \nonumber \\ % I(\data|\vectheta)
   &+& \frac{1}{2}\left(\log{Q\pi} - Q\log{2\pi}\right) \quad . % lattice
    \label{eq:objective-function}
\end{eqnarray}


\subsection{Predicting the message length of uncomputed mixtures}


Consider that the number of true components $K_{true}$ is unknown, and that we
have an optimal mixture of $K$ components in the sense that the model
parameters $\vectheta$ minimise the total message length in Eq.
\ref{eq:objective-function}.  We are interested to know whether the total
message length will decrease if we add or remove components.  For
example, our current state may be an optimized mixture with $K=2$ and we want
to know if an optimal mixture with $K=3$ will result in a shorter message
length.  By denoting $I\current$ as the message length of the current (optimal) 
mixture and $I\future$ as mixtures that have yet to be computed, we
want to know if $I\future - I\current < 0$, or rather we seek to calculate 
some probability $\textrm{Pr}(I\future - I\current < 0)$ before having to 
optimize the future mixture $I\future$.  
Taking Eq. \ref{eq:objective-function}, we can derive an expression for the 
difference in message lengths between future mixtures and the current state:
\begin{eqnarray}
  I\future - I\current &=& \Delta{K}\left[
  \left(1 - \frac{D}{2}\right)\log{2} + \frac{1}{4}\left(D(D+3)+2\right)\log\frac{N}{2\pi}\right] \nonumber \\
  &-& \log\Gamma(K+\Delta{K}) + \log\Gamma(K) + \frac{1}{2}\left(\log{Q\future} - \log{Q\current}\right) \nonumber \\
  &+& \frac{1}{2}\left(\frac{D(D+3)}{2} - 1\right)\left(\sum_{k=1}^{K+\Delta{K}}\log\weight_k\future - \sum_{k=1}^{K}\weight_k\current\right) \nonumber \\
  &-& \sum_{n=1}^{N}\log\sum_{k=1}^{K+\Delta{K}}\weight_{k}f_{k}(\data_n|\vecmean_k,\veccov_k)\future + \sum_{n=1}^{N}\log\sum_{k=1}^{K}\weight_{k}f_{k}(\data_n|\vecmean_k,\veccov_k)\current \nonumber \\
  &+& \frac{D + 2}{2}\left(\sum_{k=1}^{K}\log|\veccov_k|\current - \sum_{k=1}^{K + \Delta{K}}\log|\veccov_k|\future\right) \quad .
\end{eqnarray}




\section{Search strategy}

% Initialization

% Perturbation strategy with a twist


\section{Experimental results}

% 

% BAYES FTW

\section{Conclusion}


%\subsubsection*{Acknowledgments}

\section*{References}

References follow the acknowledgments. Use unnumbered first-level
heading for the references. Any choice of citation style is acceptable
as long as you are consistent. It is permissible to reduce the font
size to \verb+small+ (9 point) when listing the references. {\bf
  Remember that you can use a ninth page as long as it contains
  \emph{only} cited references.}
\medskip

\small

[1] Alexander, J.A.\ \& Mozer, M.C.\ (1995) Template-based algorithms
for connectionist rule extraction. In G.\ Tesauro, D.S.\ Touretzky and
T.K.\ Leen (eds.), {\it Advances in Neural Information Processing
  Systems 7}, pp.\ 609--616. Cambridge, MA: MIT Press.

[2] Bower, J.M.\ \& Beeman, D.\ (1995) {\it The Book of GENESIS:
  Exploring Realistic Neural Models with the GEneral NEural SImulation
  System.}  New York: TELOS/Springer--Verlag.

[3] Hasselmo, M.E., Schnell, E.\ \& Barkai, E.\ (1995) Dynamics of
learning and recall at excitatory recurrent synapses and cholinergic
modulation in rat hippocampal region CA3. {\it Journal of
  Neuroscience} {\bf 15}(7):5249-5262.

\end{document}
